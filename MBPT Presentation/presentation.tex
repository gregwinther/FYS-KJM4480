\documentclass{beamer}

\usepackage[utf8]{inputenc}
\usepackage[norsk]{babel}
\usepackage{amsmath}
\usepackage{bbm}
\usepackage{physics}
\usepackage{simpler-wick}

\usetheme{metropolis}

\title[MBPT]{Many-Body Perturbation Theory}

\author{Sebastian G. Winther-Larsen}

\begin{document}

\begin{frame}
	\titlepage
\end{frame}

\begin{frame}
	\frametitle{Disposisjon}
	\tableofcontents
\end{frame}

\section{Motivasjon}
\begin{frame}
	\frametitle{Mangepartikkelproblemet}
	For nesten alle systemer av interesse i kjemi kan ikke Schrödinger-likningen løses nøyaktig. 
	En trenger tilnærmede metoder, hvor en standard verktøykasse inneholder
	\begin{itemize}
	\item Hartree-Fock (HF),
	\item Konfigurasjonsinteraksjonsteori (CI),
	\item Koblet klynge-teori (CC),
	\item Perturbasjonsteori (PT) $\leftarrow$
	\end{itemize}
\end{frame}

\section{Perturbasjonteori (PT)}
\begin{frame}
	\frametitle{Formell pertubasjonsteori}
	
	Del opp 
	\begin{equation}
		\hat H = \hat H_0 + \hat V,
	\end{equation}
	slik at Schrödinger-likningen (SL) kan skrives
	\begin{equation}
		(\hat H_0 + \hat V)\ket{\Psi} = E\ket{\Psi}, \quad \ket{\Psi} = \ket{\psi_0} + \ket{\chi},
	\end{equation}
	hvor
	\begin{equation}
		\hat H_0 \ket{\Phi_0} = E^{(0)}_0\ket{\Phi_0}
	\end{equation}
	har en kjent løsning.
\end{frame}

\begin{frame}
	
	Anvender $\bra{\Phi_0}$ på Schrödingerlikningen,
	\begin{gather}
		\bra{\Phi_0}\hat H_0 \ket{\Psi} + \bra{\Phi_0}\hat V \ket{\Psi} = E\braket{\Phi_0}{\Psi} \\
		\rightarrow E  - E_0^{(0)} = \Delta E = \bra{\Phi_0}\hat V \ket{\Psi},
	\end{gather}
	Hvor jeg har brukt at $\braket{\Phi_0}{\Phi_0} = 1$, $\braket{\Phi_0}{\chi} = 0 \rightarrow \braket{\Phi_0}{\Psi} = 1$.
	
\end{frame}

\begin{frame}
	Introduserer projeksjonsoperatorer
	\begin{equation}
		\hat P = \ket{\Phi_0}\bra{\Phi_0}, \quad \hat Q = \mathbbm{1} - \hat P = \sum_{i = 1}^{\infty} \ket{\Phi_i}\bra{\Phi_i}.
	\end{equation}
	Viktige egenskaper: $\hat Q^2 = \hat Q$, $\hat P^2 =  \hat P$, $|\hat P, \hat H_0] = [\hat Q, \hat H_0] = 0$, $\hat P \hat Q = \hat Q \hat P = 0$.

	Om bølgefunksjonen skrives om en lineær kombinasjon $\ket{\Psi} = \sum_i a_i \ket{\Phi_i}$ så henter $\hat P$ ut $\ket{\Phi_0}$,
	\begin{gather}
		\hat P \ket{\Psi} = a_0\ket{\Phi_0}, \quad \hat Q \ket{\Psi} = \sum_{i\neq 0} a_i \ket{\Phi_i}\\
		\rightarrow \ket{\Psi} = \hat P \ket{\Psi} + \hat Q \ket{\Psi}.
	\end{gather}

\end{frame}

\begin{frame}

	Skriver om Schrödingerlikningen
	\begin{align}
		(\hat H_0 + \hat V)\ket{\Psi} =& E\ket{\Psi} \\
		-\hat H_0\ket{\Psi} =& (\hat V - E)\ket{\Psi} \\
		(\zeta - \hat H_0)\ket{\Psi} =& (\hat V - E + \zeta) \ket{\Psi} \\
		\hat Q (\zeta - \hat H_0) \hat Q \ket{\Psi} =& \hat Q (\hat V - E + \zeta)\ket{\Psi}.
	\end{align}

	Trenger den inverse (størrelsen) til $\hat Q (\zeta - \hat H_0) \hat Q$ i Q-rommet, også kjent som den resolvente (størrelsen) til $H_0$
	\begin{equation}
	\frac{\hat Q}{\zeta - \hat H_0} \hat Q(\zeta - \hat H_0)\hat Q = \hat Q \rightarrow \hat R \equiv \frac{\hat Q}{\zeta - \hat H_0} 
	\end{equation}

\end{frame}


\begin{frame}
	Anvender $\hat R$ på den omskrevne Schrödingerlikningen

	\begin{gather}
		\hat R \hat Q (\zeta  - \hat H_0) \hat Q \ket{\Psi} = \hat R \hat Q(\hat V - e + \zeta)\ket{\Psi} \\
		\hat Q\ket{\Psi} = \hat R(\hat V - e + \zeta)\ket{\Psi} \\
		\rightarrow \ket{\Psi} = \ket{\phi_0} + \hat R(\hat V - e + \zeta)\ket{\Psi}, 
	\end{gather}
	
	Om en betrakter dette som en rekursjonsrelasjon så får en
	\begin{equation}
		\ket{\Psi} = \sum_{m=0}^\infty [\hat R(\hat V - E + \zeta)]^m \ket{\Phi_0}.
	\end{equation}
	
	\begin{itemize}
		\item Brillouin-Wigner PT: $\zeta = E$,
		\item Rayleigh-Schrödinger PT: $\zeta = E_0^{(0)} \to -E + \zeta = - \Delta E$.  	
	\end{itemize}

\end{frame}

\section{Rayleigh-Schrödinger PT}

\begin{frame}
	\frametitle{Korreksjon ved RSPT}
	$n$-te ordens korreksjon i bølgefunksjon og energi er
	\begin{gather}
		E^{(n)} = \bra{\Phi}\hat V \ket{\Psi^{(n-1)}}\\
		\ket{\Psi^{(n)}} = \hat R \left[\hat V \ket{\Psi^{(n-1)}} - \sum_{j=1}^{n-1}E^{(n-j)}\ket{\Psi^{(j)}} \right]
	\end{gather}

\end{frame}
	

\begin{frame}
	
	\frametitle{Lavere ordens RSPT}
	Lavere ordens korreksjonsledd for energi,
	\begin{align}
		E^{(1)} =& \bra{\Phi} \hat V \ket{\Phi} \\
		E^{(2)} =& \bra{\Phi} \hat V \hat R \hat V \ket{\Phi} \\
		E^{(3)} =& \bra{\phi} \hat V \hat R \hat V \hat R \hat V \ket{\Phi} - E^{(2)}\ket{\Phi} \hat V \hat R^2 \hat V \ket{\Phi}
	\end{align}
	og bølgefunksjon
	\begin{align}
		\ket{\Psi^{(1)}} =& \hat R \hat V \ket{\Phi} \\
		\ket{\Psi^{(2)}} =& \hat R (\hat V - E^{(1)}) \hat R \hat V\ket{\Phi} \\
		\ket{\Psi^{(3)}} =& \hat R (\hat V - E^{(1)}) \hat R (\hat V - E^{(1)}) \hat R \hat V \ket{\Phi} - E^{(2)}\hat R^2 \hat V\ket{\Phi}
	\end{align}
	
\end{frame}


\begin{frame}
	
	\frametitle{Mangepartikkel-RSPT}

	Det er vanlig å partisjonere Hamiltonoperatoren i en enkeltlegemeoperator og en den perturberte delen som er en enkeltlegemeoperator pluss en tolegemeoperator
	\begin{align}
		&\hat H = \hat K + \hat L, \\
		&\hat K = \sum_p \kappa_p c^\dagger_p c_p, \\
		&\hat L = \sum_{pq} \bra{\phi_p} \hat \ell^{(1)}\ket{\phi_q}c^\dagger_p c_q + \frac{1}{4}\bra{\phi_p\phi_q}\hat \ell^{(2)} \ket{\phi_r\phi_s}c^\dagger_p c^\dagger_q c_r c_s.
	\end{align}	
	
\end{frame}

\begin{frame}
	Gunntilstandsbølgefunksjonen vi studerer er gitt ved $\ket{\Psi}$ og den uperturberte gunntillstandsfunksjonen er en Slaterdeterminant,
	\begin{equation}
		\ket{\Phi} = c_1^\dagger \dots c^\dagger_N \ket{-}.
	\end{equation}
	En eksitert tilstand kan beskrives med kvasipartikkeloperatorer
	\begin{equation}
		\ket{\Phi_X} = b^\dagger_{a_1}b^\dagger_{i_1} \dots b^\dagger_{a_{\#X}}b^\dagger_{i_{\#X}} \ket{\Phi}.
	\end{equation}
	Da må projeksjonsoperatorene bli
	\begin{equation}
		\hat P = \ket{\Phi}\bra{\Phi}, \quad \hat Q = \mathbbm{1} - \hat P = \sum_{X}\ket{\Phi_X}\bra{\Phi_X}
	\end{equation}
	Dette gir oss upertuberte energier
	\begin{equation}
		\epsilon = \bra{\Phi}\hat K\ket{\Phi} = \sum_{i=1}^N \kappa_i, \quad \epsilon_X = \bra{\Phi_X} \hat K \ket{\Phi_X} = \sum_{i=1}^N\kappa_i + \sum_{j=1}^{\#X}(\kappa_{a_j} - \kappa_{i_j}).
	\end{equation}
\end{frame}

\begin{frame}
	Den resolvente størrelsen blir
	\begin{equation}
		\hat R = \frac{\ket{\Phi_X}\ket{\Phi_X}}{\Delta\epsilon_X}, \quad \Delta\epsilon_X \equiv \epsilon - \epsilon_x = \sum_{j=1}^{\#X}(\kappa_{a_j} - \kappa_{i_j}).
	\end{equation}
	 Første- og andreordens energikorreksjon blir dermed (etter litt arbeid)
	 \begin{align}
	 E^{(1)} =& \bra{\Phi}\hat L \ket{\Phi} = \sum_i \bra{\phi_i} \hat \ell^{(1)}\ket{\phi_i} + \frac{1}{2}\sum_{ij}\bra{\phi_i\phi-j}\hat \ell^{(2)} \ket{\phi_i \phi_j}_{\text{AS}} \\
	 E^{(2)} =& \sum_{ia}\frac{\abs{\bra{\phi_a}\hat \ell^{(1)} \ket{\phi_i} + \sum_j \bra{\phi_j\phi_a} \hat \ell^{(2)}\ket{\phi_j \phi_i}}^2}{\kappa_i - \kappa_a}  + \frac{\abs{\bra{\phi_a \phi_b} \hat \ell^{(2)} \ket{\phi_i \phi_j}}^2}{\kappa_i + \kappa_j - \kappa_a - \kappa_b}
	 \end{align}
\end{frame}

\section{Møller-Plesset PT}

\begin{frame}
	\frametitle{Møller-Plesset PT}
	Vi partisjonerer den normalordnede Hamiltonoperatoren i henhold til Hartree-Fock-teori
	\begin{equation}
		\hat H_N = \hat H - \hat E_{\text{HF}} = \{\hat F \} + \{ \hat W \} = \sum_{pq}f_q^p \{ c^\dagger_p, c_q\} + \frac{1}{4}\sum_{pqrs}w^{pq}_{rs} \{c^\dagger_p c^\dagger_q c_s c_r \}
	\end{equation}
	Med antagelse om at $\hat F$ er diagonal så er 
	\begin{equation}
		\hat F\ket{\Phi} = e_0 \ket{\Phi}, \quad \hat F \ket{\Phi_X} = e_X \ket{\Phi_X}
	\end{equation}
	Videre er 
	\begin{equation}
		\{\hat F\} \ket{\Phi} = 0, \quad \{\hat F\} \ket{\Phi_X} = (e_X - E_0)\ket{\Phi_X} = \sum_j(\epsilon_{a_j} - \epsilon_{i_j})\ket{\Phi_X}
	\end{equation}
\end{frame}

\begin{frame}
	Kun den resolvente størrelsen gejnstår for å kunne regne ut energikorreksjonsleddene
	\begin{equation}
		\hat R = \sum_{X}\frac{\ket{\Phi_X}\bra{\Phi_X}}{\delta e_X}, \quad \Delta e_X = \sum_{j=1}^{\# X}(\epsilon_{i-j} - \epsilon_{a_j}).
	\end{equation}
	Det første energikorreksjonsleddet er
	\begin{equation}
		E^{(1)} = \bra{\Phi} \{\hat W \} \ket{\Phi} = 0.
	\end{equation}
\end{frame}

\begin{frame}
	Det andre energikorreksjonsleddet er litt vanskeligere å hanskes med, men starter med samme mønster som før
	\begin{equation}
		E^{(2)} = \bra{\Phi}\{\hat W\} \hat R \{\hat W\} \ket{\Phi} = \sum_X \frac{1}{\Delta e_X} \abs{\bra{\Phi}\{\hat W\} \ket{\Phi_X}}^2
	\end{equation}
	Fordi at $\hat W$ er en tolegemeoperator kan $\ket{\Phi}$ maksimalt være dobbelteksitert. En enkelteksitert tilstand vil gi null bidrag, og vi står igjen med
	\begin{align}
		\bra{\Phi} \{\hat W\}\ket{\Phi_{ij}^{aj}} = \frac{1}{4} \sum_{pqrs} w_{rs}^{pq} \bra{\Phi} \{c^\dagger_p c^\dagger_q c_r c_s \} \{b^\dagger_a b^\dagger_b b^\dagger_j b^\dagger_i  \}\ket{\Phi}
	\end{align}
	Det finnes kun én type fullstendig sammentrekning av de to operatorstrengene
	\begin{equation}
		\wick{
		\{\c4 c^\dagger_p \c3 c^\dagger_q \c2 c_r \c1 c_s \} \{ \c1 b^\dagger_a \c2 b^\dagger_b \c3 b^\dagger_j  \c4 b^\dagger_i  \}
		}	
	\end{equation}
\end{frame}

\begin{frame}
	Det finnes $2\cdot2=4$ av disse. De resulterende Kronecker-deltaene vil "drepe" summen og bytte ut enten $r$ eller $s$ med enten $a$ eller $b$, og $p$ eller $q$ med $i$ eller $j$. Dette gir
	\begin{equation}
		\bra{\Phi} \{\hat W\}\ket{\Phi_{ij}^{aj}} = w_{ab}^{ij}
	\end{equation}
	Annenordensenergikorrigeringen blir dermed
	\begin{equation}
		E^{(2)} = \sum_{i<j}\sum_{a<b} \frac{\abs{w_{ab}^{ij}}^2}{\epsilon_i + \epsilon_j - \epsilon_a - \epsilon_b}
	\end{equation}
\end{frame}

\section{Konvergensproblemer}
\section{Hybridmodeller}

\end{document}
