\documentclass{beamer}

\usepackage[utf8]{inputenc}
\usepackage[norsk]{babel}
\usepackage{amsmath}
\usepackage{bbm}
\usepackage{physics}

\usetheme{metropolis}

\title[MBPT]{Many-Body Perturbation Theory}

\author{Sebastian G. Winther-Larsen}

\begin{document}

\begin{frame}
	\titlepage
\end{frame}

\begin{frame}
	\frametitle{Disposisjon}
	\tableofcontents
\end{frame}

\section{Motivasjon}
\begin{frame}
	\frametitle{Mangepartikkelproblemet}
	For nesten alle systemer av interesse i kjemi kan ikke Schrödinger-likningen løses nøyaktig. 
	En trenger tilnærmede metoder, hvor en standard verktøykasse inneholder
	\begin{itemize}
	\item Hartree-Fock (HF),
	\item Konfigurasjonsinteraksjonsteori (CI),
	\item Koblet klynge-teori (CC),
	\item Perturbasjonsteori (PT) $\leftarrow$
	\end{itemize}
\end{frame}

\section{Perturbasjonteori (PT)}
\begin{frame}
	\frametitle{Formell pertubasjonsteori}
	
	Del opp 
	\begin{equation}
		\hat H = \hat H_0 + \hat V,
	\end{equation}
	slik at Schrödinger-likningen (SL) kan skrives
	\begin{equation}
		(\hat H_0 + \hat V)\ket{\Psi} = E\ket{\Psi}, \quad \ket{\Psi} = \ket{\psi_0} + \ket{\chi},
	\end{equation}
	hvor
	\begin{equation}
		\hat H_0 \ket{\Phi_0} = E^{(0)}_0\ket{\Phi_0}
	\end{equation}
	har en kjent løsning.
\end{frame}

\begin{frame}
	
	Anvender $\bra{\Phi_0}$ på Schrödingerlikningen,
	\begin{gather}
		\bra{\Phi_0}\hat H_0 \ket{\Psi} + \bra{\Phi_0}\hat V \ket{\Psi} = E\braket{\Phi_0}{\Psi} \\
		\rightarrow E  - E_0^{(0)} = \Delta E = \bra{\Phi_0}\hat V \ket{\Psi},
	\end{gather}
	Hvor jeg har brukt at $\braket{\Phi_0}{\Phi_0} = 1$, $\braket{\Phi_0}{\chi} = 0 \rightarrow \braket{\Phi_0}{\Psi} = 1$.
	
\end{frame}

\begin{frame}
	Introduserer projeksjonsoperatorer
	\begin{equation}
		\hat P = \ket{\Phi_0}\bra{\Phi_0}, \quad \hat Q = \mathbbm{1} - \hat P = \sum_{i = 1}^{\infty} \ket{\Phi_i}\bra{\Phi_i}.
	\end{equation}
	Viktige egenskaper: $\hat Q^2 = \hat Q$, $\hat P^2 =  \hat P$, $|\hat P, \hat H_0] = [\hat Q, \hat H_0] = 0$, $\hat P \hat Q = \hat Q \hat P = 0$.

	Om bølgefunksjonen skrives om en lineær kombinasjon $\ket{\Psi} = \sum_i a_i \ket{\Phi_i}$ så henter $\hat P$ ut $\ket{\Phi_0}$,
	\begin{gather}
		\hat P \ket{\Psi} = a_0\ket{\Phi_0}, \quad \hat Q \ket{\Psi} = \sum_{i\neq 0} a_i \ket{\Phi_i}\\
		\rightarrow \ket{\Psi} = \hat P \ket{\Psi} + \hat Q \ket{\Psi}.
	\end{gather}

\end{frame}

\begin{frame}

	Skriver om Schrödingerlikningen
	\begin{align}
		(\hat H_0 + \hat V)\ket{\Psi} =& E\ket{\Psi} \\
		-\hat H_0\ket{\Psi} =& (\hat V - E)\ket{\Psi} \\
		(\zeta - \hat H_0)\ket{\Psi} =& (\hat V - E + \zeta) \ket{\Psi} \\
		\hat Q (\zeta - \hat H_0) \hat Q \ket{\Psi} =& \hat Q (\hat V - E + \zeta)\ket{\Psi}.
	\end{align}

	Trenger den inverse (størrelsen) til $\hat Q (\zeta - \hat H_0) \hat Q$ i Q-rommet, også kjent som den resolvente (størrelsen) til $H_0$
	\begin{equation}
	\frac{\hat Q}{\zeta - \hat H_0} \hat Q(\zeta - \hat H_0)\hat Q = \hat Q \rightarrow \hat R \equiv \frac{\hat Q}{\zeta - \hat H_0} 
	\end{equation}

\end{frame}


\begin{frame}
	Anvender $\hat R$ på den omskrevne Schrödingerlikningen

	\begin{gather}
		\hat R \hat Q (\zeta  - \hat H_0) \hat Q \ket{\Psi} = \hat R \hat Q(\hat V - e + \zeta)\ket{\Psi} \\
		\hat Q\ket{\Psi} = \hat R(\hat V - e + \zeta)\ket{\Psi} \\
		\rightarrow \ket{\Psi} = \ket{\phi_0} + \hat R(\hat V - e + \zeta)\ket{\Psi}, 
	\end{gather}
	
	Om en betrakter dette som en rekursjonsrelasjon så får en
	\begin{equation}
		\ket{\Psi} = \sum_{m=0}^\infty [\hat R(\hat V - E + \zeta)]^m \ket{\Phi_0}.
	\end{equation}
	
	\begin{itemize}
		\item Brillouin-Wigner PT: $\zeta = E$,
		\item Rayleigh-Schrödinger PT: $\zeta = E_0^{(0)} \to \-E + \zeta = - \Delta E$.  	
	\end{itemize}

\end{frame}

\section{Rayleigh-Schrödinger PT}
\section{Møller-Plesset PT}
\section{Konvergensproblemer}
\section{Hybridmodeller}

\end{document}
